\chapter{Conclusioni}


I sistemi informatici complessi vengono adoperati sempre pi� frequentemente per svolgere compiti altamente critici che coinvolgono la sicurezza e l'incolumit� delle persone. E' pertanto
di fondamentale importanza
che vengano adottate delle tecniche in grado di mantenere sempre corrette le funzioni per le quali sono
destinate.\\ In ambito informatico il concetto di difesa � sempre pi� rilevante, infatti riuscire a possedere una solida difesa permette di tutelare i dati da attacchi esterni e di preservare i sistemi dai danni inerenti l'integrit�, la disponibilit� e la confidenzialit� delle informazioni in essi custodite. Attraverso l'adozione di soluzioni hardware e software � possibile raggiungere un elevato grado di protezione; tali soluzioni innalzano il livello di difesa, ma non possono garantire
la totale protezione dei sistemi informatici, infatti le tecniche sviluppate dagli attaccanti per ottenere le informazioni desiderate, diventano sempre pi�
mirate ed esperte. Quindi, � di fondamentale importanza rimanere sempre aggiornati sulle nuove tecniche di attacco e continuare a indagare sulle vulnerabilit� dei software e dei protocolli di comunicazione, con lo scopo di riuscire a anticipare le mosse dell'attaccante.\\ Concludendo, nel lavoro appena descritto riguardante l'apparato IMR, � stato possibile osservare che
\textit{l'hazard analysis rappresenta una delle operazioni fondamentali per la prevenzione delle criticit�} in un
sistema informatico. Attraverso le parole guida delineate dalla procedura � stato possibile individuare tutti i potenziali pericoli e i problemi di operativit� che possono portare a condizioni di funzionamento non conformi. Grazie a questa procedura sono state trovate tutte le mitigazioni necessarie ad affrontare le situazioni critiche a cui il sistema potrebbe andare incontro. Infatti, tramite l'utilizzo di un canale alternativo per confermare l'esecuzione del comando e l'adozione del protocollo PVS, un operatore � in grado di effettuare tutte le attivit� di manutenzione in totale sicurezza. Tale analisi, oltre a migliorare le condizioni di sicurezza del sistema, pu� anche essere utilizzata in futuro per essere aggiornata e considerata un punto di partenza per effettuare ulteriori analisi ancor pi� dettagliate.









