\chapter{Introduzione}
L'Information Technology (IT) svolge un ruolo fondamentale nella vita di tutti i giorni, nel modo in cui riusciamo a relazionarci, a lavorare, a vivere la nostra quotidianeit�. L'IT permea quasi tutti i nostri comportamenti quotidiani e pu� essere considerata lo strumento pi� efficace per la creazione di ricchezza economica da parte delle aziende.\\ Grazie all'IT infatti le aziende hanno potuto generare vantaggi e sfruttare nuove opportunit�. Oggi questi vantaggi e benefici, comprensivi di applicazioni, funzionalit� e infrastrutture, sono alla portata di tutti.\\ Visto l'importante ruolo assunto dall'IT, nasce l'esigenza di preservare i sistemi informatici; a questa necessit� fa capo l'\textit{hazard analysis}, o analisi dei rischi, un processo di valutazione delle criticit� inerenti un sistema informatico.\\
Il \textbf{rischio} � la potenzialit� che un'azione o un'attivit� porti a una perdita o ad un evento indesiderabile. Quello del rischio � un concetto connesso con le aspettative umane e la loro capacit� di predizione/intervento in situazioni non note o incerte. L'uomo nel corso degli anni ha imparato a confrontarsi con il rischio con un atteggiamento di sfida, ricercando sempre pi� un equilibrio tra razionalizzazione degli eventi e utilizzo dell'intuito. Nel linguaggio comune, rischio � spesso usato come sinonimo di probabilit� di una perdita o di un pericolo/minaccia. In generale, ogni indicatore di rischio � proporzionale all'effetto atteso e alla sua probabilit� di accadimento. \cite{quattordici}\\ L'esigenza di iniziare a quantificare i rischi � nata in ambito bancario, infatti fin dal tardo medioevo i banchieri sono abituati a gestire il rischio di credito, ovvero il principale rischio a cui essi sono esposti. I banchieri lombardi che a partire dal 1100 operavano in Francia, Germania ed Inghilterra utilizzavano gi� efficaci tecniche di mitigazione del rischio di credito, quali ad esempio la richiesta di cessione in pegno di oggetti di valore.\cite{otto} \\ 
Non � possibile individuare un comportamento umano o un'attivit� naturale che non venga sottoposta a rischi temporanei o costanti; solamente le scienze pure, come la matematica o la fisica,  governate dalle ferree leggi del puro determinismo, non si confrontano con tale concetto.\\


\section{Scopo della tesi}
Gli obiettivi proposti nel lavoro di tesi sono quelli di riportare l'importanza dell'\textit{Hazard analysis} in ambito ferroviario e di applicare tale procedura all'apparato Interfaccia Mobile Remota (IMR) di RFI. Tale apparato nasce con lo scopo di migliorare e velocizzare l'esecuzione delle operazioni di manutenzione lungo le linee ferroviarie. Tramite l'IMR deve diventare possibile per un operatore, tramite un tablet, effettuare una serie di operazioni, come impedire il passaggio di treni in una zona della linea ferroviaria, senza recarsi fisicamente in stazione per ottenere l'autorizzazione. Poich� il malfunzionamento di una qualunque delle fasi dell'operazione � critica per la vita dell'operatore e per il funzionamento della linea, tale procedura remota deve essere sicura, quindi eventuali malfunzionamenti devono essere visibili per l'operatore. Grazie alla procedura di hazard analysis � stato possibile evidenziare le principali criticit� del sistema in questione e quindi ricercare una soluzione oppurtuna per ognuna di esse.