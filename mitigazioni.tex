\chapter{Mitigazioni}

\subsection{Mitigazioni}
Le mitigazioni alle quali ho pensato per risolvere gli azzardi rilevati sono le seguenti:
\begin{itemize}
	\item \textit{avere a disposizione un tablet di riserva}: questo accorgimento � necessario quando il tablet utilzzato dall'operatore non riesce a eseguire correttamente le operazioni di base (touchscreen non reattivo, spegnimento improvviso del tablet, mancata gestione della batteria ecc..);
	\item \textit{riaggiornamento pagina}: da effettuare quando ad esempio la lista delle stazioni non � completa;
	\item \textit{utilizzo autenticazione biometrica}: l'autenticazione, come gi� anticipato, avviene tramite password e un meccanismo di chiave asimmetrica: un utente quindi per accedere al Terminale Operatore deve digitare prima il suo nome utente, la sua password e indicare la chiave privata. Questo semplice approccio di autenticazione pu� essere preso di mira per effettuare vari tipi di attacco, come � riportato nella tabella. L'utente pu� inoltre dimenticare la password e quindi non riuscire ad accedere al sistema. Un meccanismo che permettere di risolvere tutte le problematiche sopra elencate � l'adozione di un meccanismo di autenticazione biometrica. Attraverso questo approccio un individuo viene autenticato, basandosi su pattern di riconoscimento, attraverso le sue caratteristiche fisiche (es. geometria della mano, impronta digitale, timbro della voce ecc...);
	
	\item \textit{utilizzare password potenti}: per fare in modo che il meccanismo di autenticazione tramite password sia solido e sia difficile per un attaccante introdursi nel sistema � necessario che la password adottata sia pi� complessa possibile. E' quindi consigliabile utilizzare sia lettere maiuscole che minuscole, segni e numeri;
	
	\item \textit{numero tentativi autenticazione limitati}: � possibile che un attaccante per accedere al sistema abbia intenzione di provare tutte le password di un dizionario. Per evitare che tale attacco sia possibile, � necessario imporre che i tentativi di autenticazione siano limitati, ad esempio un numero appropriato di tentativi di autenticazione potrebbe essere pari a 3;
	
	\item \textit{modificare la password periodicamente}: per garantire la sicurezza del sistema � opportuno che le passoword degli addetti manutenzione vengano modificate periodicamente;
	
	\item textit{memorizzare la chiave privata in sicurezza}: per evitare che il file contente la chiave privata venga rubato da individui non autorizzati ad accedere al sistema � opportuno che venga salvato in una posizione di massima sicurezza;
	
	\item CONNESSIONE DEBOLE????? che famo???
